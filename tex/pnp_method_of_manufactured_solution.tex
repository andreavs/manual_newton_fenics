% LaTeX file for a 1 page document
\documentclass[12pt]{article}

\title{Implementation of the Poisson-Nernst-Planck equations in FEniCS}




\usepackage{amsmath}
\usepackage{helvet}
\usepackage{fullpage}
\usepackage{amsthm}
\usepackage{amsfonts}
\usepackage{graphicx}
\usepackage{caption}
\usepackage{subcaption}
\usepackage[english]{babel}
\usepackage[T1]{fontenc}
%\usepackage{subfigure}
\usepackage{epstopdf}
\epstopdfsetup{update}
\usepackage[hyphens]{url}
\usepackage{gensymb}
\usepackage{verbatim}
%\usepackage{slashed}
\usepackage{amssymb}
\usepackage{amsfonts}
\usepackage[]{units}
\usepackage{verbatim}
\usepackage{cprotect}
\usepackage{cleveref}

\newcommand{\Jvec}{\textbf{J}}
\newcommand{\Ivec}{\textbf{I}}
\newcommand{\xvec}{\textbf{x}}
\newcommand{\yvec}{\textbf{y}}
\newcommand{\Na}{\text{Na}}
\newcommand{\K}{\text{K}}
\newcommand{\Cl}{\text{Cl}}
\newcommand{\nvec}{\textbf{n}}

\newcommand{\dd}{\text{d}}

\date{\today}
\author{Andreas}
\begin{document}
\maketitle

\begin{abstract}
We will show how to implement the Poisson-Nernst-Planck (PNP) equations in FEniCS. 
\end{abstract}

\section{Introduction}
We will study the dynamics of ions in a medium. We will consider a system of two ions. One positive ion with valency, indexed by 1, and one negative ion with valency -1, indexed by 2. We can think of the positive ion as potassium and the negative ion as chloride, although the spesific ion types will not be important in this text

\section{The equations}
The ion concentration dynamics are influenced by two principles: 
\begin{enumerate}
\item The ions will move by diffusion, from areas of with high concentration to areas of low concentration. 
\item As the ions are charged, they will be influenced if there is an electric field present. For the same reason, the ions themselves will create an electric field. 
\end{enumerate}
We assume that these concentration currents are additive. Starting with the continuity equation, we get
\begin{equation}
{\partial c_i \over \partial t} = - \nabla \cdot \Jvec_i + f_i,
\end{equation}
where $c_i$ is the concentration of ion type $i$, $\Jvec_i$ is the concentration current of ion type $i$, and $f_i$ is a source term
\begin{align}
\Jvec_i & = \Jvec_i^\text{diff} + \Jvec_i^{\text{field}}, \\ 
 &= -D_i\nabla c_i - {D_iz_i \over \psi} c_i \nabla \phi,
\end{align}
where $D_i$ is the diffusion coefficient, $z_i$ is the valency of ion type $i$, $\psi$ is a physical constant, and $\phi$ is the electric field. This equation is called the \textit{Nernst-Planck} equation. The electric field is found by the Poisson equation: 
\begin{equation}
\nabla^2 \phi = -{\rho \over \epsilon} = -{F \over \epsilon} \sum_i z_i c_i,
\end{equation}
where $\rho$ is the charge concentration, $F$ is Faradays constant, $\epsilon$ is the permittivity of the medium (in this case the extracellular space). Together, these equations form the \textit{Poisson-Nernst-Planck} (PNP) system of equations. We present the full set of equations: 

\begin{align}
{\partial c_1 \over \partial t} & = \nabla \cdot \left[ D_1\nabla c_1 + {D_1z_1 \over \psi} c_1 \nabla \phi \right] + f_1 \\ 
{\partial c_2 \over \partial t} & = \nabla \cdot \left[ D_2\nabla c_2 + {D_2z_2 \over \psi} c_2 \nabla \phi\right] + f_2 \\ 
\nabla^2 \phi & =-{F \over \epsilon} (z_1c_1 + z_2c_2)\label{eq:phi}
\end{align}

\section{Boundary Conditions}

We will use Dirichlet boundary conditions for the concentrations, and a pure von Neumann boundary condition for the field: 
\begin{align}
c_1 & = c_1^D & \text{on } \partial \Omega \\
c_2 & = c_2^D & \text{on } \partial \Omega \\
\nabla \phi \cdot \nvec & = 0 & \text{on } \partial \Omega 
\end{align}

 
\section{Units and dimensions}
Note that in terms of natural constants, $\psi = RT/F$, where $R$ is the gas constant, $F$ is Faradays constant and $T$ is the temperature.
\begin{center}
 \begin{tabular}{c|l|l}
 symbol & explanation & units \\
 \hline
 $x$ & position & $\mu$m \\
 $t$ & time & ms \\
 $\phi$ & potential & V \\
 $c_i$ & concentration & mmol/ml \\ 
 $D_i$ & diffusion coefficient & $\mu$m$^2$/ms \\
 $f_i$ & source term & mmol/(ml$\times$ms) \\
 $z_i$ & valency & (none) \\
 $\psi$ & see text & J/C \\
 $F$ & Faradays Constant & C/mol \\
 $\epsilon$ & permittivity & pF/m \\
 \end{tabular}
 \end{center}

\section{FEniCS implementation}

\subsection{Time discretization}

\subsection{Weak form}

\section{Method of Manufactured Solution}
In order to test the implementation, we use the method of manufactured solution. We set the solution of the system of equations, and then modify the source term to make the chosen solution correct. Since we only have two source terms, we can only freely choose the solution of two of the equations, and then chose the last one so that the solutions are consistent. If we set the solution for $c_1$ and $\phi$, then we can find the solution of $c_2$ from Equation \eqref{eq:phi}. We also have to set $\phi$ so that it is consistent with the boundary conditions, as well as the additional condition $\int_\Omega \phi\, \text{d}x = 0$. We set the following set of solutions:
\begin{align}
c_1 &= \cos^3(x) \sin(t) \\ 
\phi &= (\sin^2(\pi x) - 0.5)\cos^2(t) \\
c_2 & = -{1\over z_2} \left( {F \over \epsilon}\nabla^2\phi + z_1c_1\right)
\end{align}







\end{document}